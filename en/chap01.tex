\chapter{Introduction}

\section{Introduction to the problematic}
Voice control or communication is a common feature of many systems nowadays.
Its applications ranges from simple one-word control commands to complex communication in spoken dialogue systems.
In this work, we consider mainly these complex systems.
For the sake of clarity, we now briefly describe setting of such system.
It usually contains Automatic Speech Recognition (ASR) module, so the natural speech can be recognized and translated into words.
The system then somehow derives an appropriate response, typically in the form of sentence written in natural language.
This response can be displayed in the written form, however, it is more common to generate an audio with human voice reading the response.
Although it is possible to use only a limited set of prerecorded utterances, it is desired to be able to read an arbitrary phrase.
One reason is, that it may be difficult to read named entities and numerical values such as time and date.
Also, the usage of variable utterances provides better user experience.
Because of this, a Text-To-Speech (TTS) module is usually also part of dialogue systems.
The purpose of this module is to transform a (generally arbitrary) written text to audio file containing the utterance read in natural voice.
Modern TTS systems produce audio waveforms that sound quite naturally and the pronunciation is relatively good.
Nevertheless, it have difficulties when it comes to  so called Out-Of-Vocabulary (OOV) words.
That is because the system is usually trained using certain set of words, typically from one language.
But real applications often require to pronounce named entities or other language- or domain- specific words, that cannot be present during the training phase.
This can cause situations, when words are mispronounced.
Although it does not happen often, the negative effect can be quite strong, since it is inconvenient for the user when his or her name is pronounced with mistakes.
\linebreak\linebreak
In this work, we aim to improve the TTS system pronunciation of desired words.
To achieve this, we employ feedback gathered from the user.
That means, we allow user to provide better example of pronunciation and thus we can correct ours.
We are able to improve the TTS system by processing the obtained recording, deriving a phonetic transcription (i.e. pronunciation) and adding it to the TTS vocabulary.
Moreover, the derived pronunciations can be used to improve the recognition ability of the ASR module.
Also, we propose algorithm that can identify words, that are potentially difficult to pronounce without any prior language knowledge.
More details regarding this issue can be found in respective sections.
As it has been suggested, our method has got potentially very useful applications.
It can be used to enlarge vocabulary of TTS or ASR systems both offline or on the fly using the live user's feedback.
There exist several ways how to obtain such a feedback, however, this is not a subject of this work.
Theoretically, the method can work with just one gold example, however, it is generally possible to ask user two or three times while not bothering him too much. (TODO: source)
In \figref{dialogsample} we provide basic example of simple dialogue, illustrating the process.\pagebreak
\begin{center}
\begin{figure}[h]
\texttt{System: Hello, /AANDRZHEZH/.\linebreak
User: You said it wrong, my name is /ONDRZHEI/.\linebreak
System: /ANDREY/, correct?\linebreak
User: No, it is /ONDRZHEI/.\linebreak
System: Oh, /ONDRZHEI/?\linebreak
User: That's right.
}
\caption{Sample dialogue illustrating the pronunciation correction. The transcriptions of the user's name are given in ARPABET\cite{Arpabet}}
\label{dialogsample}
\end{figure}
\end{center}
Please note, that the dialogue policy is not part of this work and the way, how the feedback is obtained depends on the respective dialogue system.

\section{Overview of used techniques}
\subsection{Audio signal processing}
\subsection{Finite state transducers\cite{mohri1997finite}}
Finite state transducers (FSTs) are finite state automata that are augmented by addition of output labels to each transition.
We can further modify the FST and add a weight to each edge.
We define a semiring $(S,\oplus,\otimes,\overline{0},\overline{1})$ as a system that fullfills the following conditions:
\begin{itemize}
\item $(S,\oplus,\overline{0})$ is a commutative monoid with identity element $\overline{0}$
\item $(S,\otimes,\overline{1})$ is a monoid with identity element $\overline{0}$
\item $\otimes $ distributes over $\oplus$
\item $\forall a \in S: a \otimes \overline{0} = \overline{0} \otimes a = \overline{0}$
\end{itemize}
We this definition of the semiring, we can describe a weighted FST\cite{mohri2009weighted} $T$ over a semiring $S$ formally as a 8-tuple $(\Sigma,\Delta,Q,I,F,E,\lambda,\rho)$ where $\Sigma, \Delta$ are finite input and output alphabets, $Q$ is a finite set of states, $I \subset Q$ is a set of initial states, $F \subset Q$ is a set of final states, $E$ is a multiset of transitions, $\lambda: I \rightarrow S$ is an initial weight function and $\rho: F \rightarrow S$ is a final weight function.
Each transition is element of $Q \times \Sigma \times \Delta \times S \times Q$.
Note, that since $E$ is a multiset, it allows two transitions between states $p$ and $q$ with the same input
and output label, and even the same weight.
However, this is not used in practice.
An example of weighted FST is given in TODO.
Finite state transducers have been used widely in many areas, especially linguistics.
They can represent local phenomena encountered in the study of language and they usage often leads to compact represantations.
Moreover, it is also very good from the computational point of view, since it advantageous in terms of time and space efficiency.
Whole area of algorithms has been described that consider FSTs.
One of the most important is $determinization$, which determinize paths in the FST and thus allows the time complexity to be linear in the length of input.
Weighted FSTs have been widely used in the area of automatic speech recogniton.
The allows to compactly represent the hypothesis state of the acoustic model as well as combining it with the information contained in the language model.
\subsection{Automatic speech recognition (ASR)}
\subsubsection*{Overview}
The task in ASR is quite clear: An utterance spoken in natural language should be translated into its text representation.
Formally, we are given a sequence of acoustic observations $\textbf{X} = X_1X_2\dots X_n$ and we want to find out the corresponding word sequence $\textbf{W} = W_1W_2\dots W_m$ that has a maximum posterior probability $P(W | X)$i.e., according to the Bayes rule-:
\begin{center}
\begin{equation}
\hat{\textbf{W}} = argmax_w \frac{P(\textbf{W})P(\textbf{W}|\textbf{X})}{P(\textbf{X})}
\end{equation}
\end{center}
The challenge consists of two parts.
One, we have to build accurate acoustic model that describes the codnditional distribution $P(W|X)$.
Second part is to create a language model that reflects the spoken language tah should be recognized.
The decoding process of finding the best matched word sequence
W to match the input speech signal X in speech recognition systems is more than a simple
pattern recognition problem, since there is an infinite
number of word patterns to search in continuous speech recognition.
Because the speech signal is continuous, it is transformed to discrete sequence of samples.
MFFC's are commonly used to represent the signal.
The process of deriving this sequence is described in further detail in the signal processing chapter.
In general, it is difficult to use whole-word models, because every new task may contain unseen words.
Even if we had sufficient number of examples to cover all the words, the data would be too large.
Thus, we have to select basic units to represent salient acoustic and phonetic information.
These units should be $accurate, trainable$ and $generalizable$.
It turns out that the most appropriate units are phones.
Phones are very good for training and generalization, however, it does not include contextual information.
Because of this, so called triphones are commonly used.
To model the acoustic, Hidden Markov Models are commonly used, because it can deal with unknown alignments.
In recent years, neural networks have experienced big breakthrough and it found application even in the area of speech recognition.
Namely, recurrent neural networks are able to work with an abstraction of memory and process sequences of variable length, so it overcomes the HMM's in terms of accuracy.
The language and acoustic models are traditionally trained independently and they are joined during the decoding process.
A special graph is built in order to make the decoding.
Its nodes represents acoustic units and edges are assigned costs.
This assignment corresponds to the likelihood determined by the acoustic model as well as the language model.
Decoding the output is realized as searching the best path through this graph.

The decoding graph is costructed in such a way, that the path can contain only words present in the vocabulary the language model is built on.
Thus, each possible path consists of valid words.
Because of that, we can see the word as a basic unsplittable unit, although it is actually composed from phones, or triphones respectively.
However, we can build a lattice that allows to construct the path from phones and thus recognize the input on the phonetic level.
Also the search algorithm preserves alternative paths so in the end it gives us not only the best path but also list of alternative hypotheses.
We call it the nbest list.
Each hypothesis in the nbest list is associated with its likelihood that expresses information from both the language and acoustic model.
In chapter \td{chapter} we explore the nbest list and propose method that exploits it.
\subsection{Text-to-speech (TTS)}
\subsubsection*{Overview\cite{taylor2009text}}
The text-to-speech problem can be looked at as a task of transforming an arbitrary utterance in natural language from its written form to the spoken one.
In general, these two forms have commonalities in the sense, that if we can decode the form from the written signal, then we have virtually all the information we require to generate a spoken signal.
Importantly, it is not necessary to (fully) uncover the meaning of the written signal, i.e. employ Spoken Language Understanding (SLU).
On the other hand, we may need to generate prosody which adds some information about emotional state of the speaker or emphasizes certain parts of the sentence and thus changing the meaning slightly.
To obtain prosodic information, sophisticated techniques need to be involved, since its not a part of common written text, except some punctuation.
Another difficulties stems from the fact, that we often need to read numbers, or characters with special meanings such as dates or mathematical equations.
The problem of converting text into speech has been heavily explored in the past and the TTS systems (engines) are very sophisticated nowadays.
The subsequent section, describes briefly the typical architecture of TTS system.
\subsubsection*{TTS system architecture\cite{taylor2009text}}
Let us now describe a common use of TTS system.
The architecture usually consists of several modules, although some end-to-end systems base on neural networks have appeared recently\cite{van2016wavenet} \cite{wang2017tacotron}.
First, the input text is divided into sentences and each sentence is further tokenized, based on whitespace characters, punctuation etc.
Then, the non-natural language tokens are decoded and transformed into text form.
We then try to get rid of ambiguity and do prosodic analysis.
Although much of the information we might ideally like is missing from the text, we use algorithms to determine the phrasing, prominence patterns and tuning the intonation of the utterance.
Next is the synthesis phase. The first stage in the synthesis phase is to take the words we have just found and encode them as phonemes.
There are two main approaches how to deal with the synthesis phase.
More traditional approach is so called \textbf{unit concatenation}.
This approach uses a database of short prerecorded segments of speech (roughly 3 segments per phoneme).
These segments are then concatenated together with some use of signal processing so they fit together well.
An alternative is to use statistical, machine learning techniques to infer the specification-to-parameter mapping from data.
While this and the concatenative approach can both be described as data-driven, in the concatenative approach we are effectively memorizing the data, whereas in the statistical approach we are attempting to learn the general properties of the data.
Trained models are able to transform the phonemes into audio signal representation.
This representation is then synthesized into waveforms using a vocoder module.
Although unit concatenation approaches generally achieve better results, the statistical ones are more flexible and have good possibilities to fine tuning or postprocessing.
\subsubsection*{Grapheme to Phoneme conversion and its drawbacks}
In the stage of converting graphemes (i.e. text) to phonemes, the g2p module is usually used.
The g2p task is to derive pronunciations from ortographic transcription.
Traditionally, it was solved using decision trees models.
It can also be formulated as a problem of translating one sequence to another, so neural networks can be used to solve this task\cite{yao2015sequence}.
However, in our work we use joint-sequence model\cite{bisani2008joint}, which estimates joint distribution of phonemes based on probabilities derived from n-grams.
The g2p module is crucial component of the system and it has great impact on the final pronunciation.
Since we use machine learning techniques and train on the dictionary data, the model inherently adopts pronunciation rules from the respective language.
Although it is in principle possible to  train multilingual g2p\cite{schlippe2012grapheme}, it is usually not the case in common TTS systems.
The problem arises, when words that originate in some foreign language should be pronounced.
In this work we address this issue by deriving pronunciations from the ASR output.
\subsubsection*{Speech Synthesis Markup Language (SSML)\cite{taylor1997ssml}}
SSML is a standard, that allows to specify aspects of the speech.
It is an input to the synthesis module and many TTS engines support its use.
We can specify emotions, breaks etc. with the help of SSML.
More importantly, it can be used to input particular phonemes and thus circumvent the g2p module.
In our work we use this method to feed our phonetic transcriptions into the engine.
The disadvantage of this method is, that many TTS engines process the SSML input imperfectly or not at all.
However, in principle it is possible to add the transcriptions directly into the system's vocabulary.
\subsubsection*{Overview of the used TTS systems}
\begin{enumerate}
\item Cereproc TODO:citations
This engine is a commercial software that is free for educational purposes.
Although it is not open-sourced, we use it for its good quality and reliable outputs.
\item MaryTTS
MaryTTS is a German open-source project written in Java.
It is highly customizable since it is modular.
Thus we can explore output of an arbitrary module or replace it in the processing pipeline.
MaryTTS is based on client-server architecture.
\item gTTS is a TTS service provided online by Google.
It achieves very good quality, however it allows ony use of one voice in the free version.
\item Pico
SVOX Pico TTS is a lightweight engine that lacks a good quality of the gTTS, however it offers a large selection of voices and it is commonly used on the phones with Google's Android operating system.

\end{enumerate}
\subsection{Algorithms description}
\subsubsection*{Dynamic Time Warping}\cite{ratanamahatana2004everything}
The measurement of similarity between two time series is
an important subroutine in many applications.
Moreover, sometimes we need to align two sequences that describe same data but are of different lengths.
Both this tasks can be solved with use of DTW.
Suppose we have two time series, a sequence $Q$ of length $n$,
and a sequence $C$ of length $m$, where
\begin{center}
$Q = q_1 ,q_2 ,\dots, q_i ,\dots ,q_n$
\linebreak
$C = c_1 ,c_2 ,\dots, c_j \dots, c_m$
\end{center}
To align these two sequences using DTW, we first
construct an n-by-m matrix where the $(i^{th} , j^{th} )$ element of
the matrix corresponds to the squared distance, $d(q_i , c_j) =
(q_i – c_j )$ , which is the alignment between points $q_i$ and $c_j$.
To find the best match between these two sequences, we
retrieve a path through the matrix that minimizes the total
cumulative distance between them.
In particular, the optimal path is the path that minimizes the
warping cost:
\begin{center}
$DTW ( Q , C ) = \sqrt[]{\sum_{k=1}^{K}w_k}$
\end{center}
where $w_k$ is the matrix element $(i,j)_k$ that also belongs to $k^{th}$
element of a warping path $W$, a contiguous set of matrix
elements that represent a mapping between $Q$ and $C$.
Methods of dynamic programming are used to fill in the values and find alignment of sequences.
Thus a result of this algorithm is a mapping $\sigma$ that can be used to construct sequence of pairs $A$ containing members of both sequences $C,Q$ in each time step.
$\sigma$ is constructed in such a way, that it holds:
\begin{center}
$A_i = (C_{\sigma (C,i)}, Q_{\sigma (Q, i)})$ \linebreak
$i = 1 \dots K$ \linebreak
$max(|C|,|Q|) \le K$ \linebreak
and $\sum_{i=0}^K d(\sigma (C,i), \sigma(Q,i)$ is minimal
\end{center}
Where $d(x,y)$ is an arbitrary distance metric.
In our application, we want to align two sequences of phonemes, so we choose Levensthein distance to be our metric.
\section{Related Work}
\subsection{Grapheme-to-phoneme conversion}
An automatic grapheme-to-phoneme conversion was first considered in the context of TTS applications. The input text needs to be converted to a sequence of phonemes which is then fed into a speech synthesizer.
It is common in TTS systems that they first try to find the desired word in the dictionary and if it doesn't find it, it employs the grapheme-to-phoneme ($g2p$) module.
A trivial approach is to employ a \textit{dictionary look-up}.
However, it cannot handle context and inherently covers only finite set of combinations.
To overcome this limitations, the rule-based conversion was developed. Kaplan and Kay \cite{kaplan1994regular} formulate these rules in terms of finite-state automata.
This system allows to greatly improve coverage.
However the process of designing sufficient set of rules is difficult, mainly since it must capture irregularities.
Because of this, a \textit{data-driven} approach has to be introduced.
Many machine learning techniques was explored, starting with Sejnowski and Rosenberg \cite{sejnowski1988nettalk}.
The approaches can be divided into three groups.
\subsubsection{Techniques based on local similarities}
The techniques presuppose an alignment of the training data
between letters and phonemes or create such an alignment in a separate preprocessing
step.
The alignment is typically construed so that each alignment
item comprises exactly one letter. Each slot is then classified (using its context) and a correct phoneme is predicted.
\subsubsection{Pronunciation by analogy}
These methods search for local similarities in the training lexicon.
So it can be understood as a variation of the \textit{nearest-neighbors} approach.
\subsubsection{Probabilistic approaches}
The problem can be looked at from a probabilistic point of view. One way to do this is to employ so called \textit{joint sequence models} \cite{bisani2008joint}. We use an open-source implementation of such a model in our work. This approach formalizes the task as follows:
\begin{equation}
\boldsymbol{\varphi(g)} = \argmax_{\varphi'\in\Phi^*} p(\boldsymbol{g,\varphi'})
\end{equation}
where * denotes a Kleene star. In other words, for a given ortographic form $\boldsymbol{g} \in G^*$ we want to find the most likely pronunciation $\boldsymbol{\varphi} \in \Phi^*$, where $G, \Phi$ are ortographic and phonetic alphabets.
\linebreak \linebreak
Generally, the problem of the wrong pronunciation in TTS  is caused by a bad phonetic transcription.
Traditional TTS systems are modular, one module's output is inputted into the next one.
Because of this fact, the errors cumulate and thus the mistakes made by $g2p$ cannot be repaired.
So if we want to improve the pronunciation, we can try to improve the $g2p$ as it is done in \cite{deri2016grapheme}.
Authors in this work propose a method of exploiting a $g2p$ trained on a language with a high number of available resources to create a $g2p$ for language for which we do not have sufficient number of examples.
This method relies on the existence of a conversion mapping between these two languages.
Also it requires to do the conversion for every new language.
\subsection{Learning pronunciation from spoken examples}. 
In theory, a model could be created which accepts an ortographic form together with language information and outputs a correct transcription.
However, this information is typically not available.
Also, if we want to learn a new pronunciation of just one word, it's more convenient to do it in a different way.
Authors of \cite{slobada1996dictionary} introduce method of deriving correct pronunciation for a word.
They argue that in the spontaneous speech, the most frequent pronunciation doest not need to be the correct one.
Thus they allow multiple way how to pronounce certain words.
The pronunciations are obtained using n-best list of the modified recognizer.
\cite{mcgraw2013learning} and \cite{reddy2011learning} improve this approach.
Both these works modify the recognizer's training procedure and use an EM algorithm to estimate the parameters.
They also introduce the concept of \textit{graphones} - units consisting of pairs (\textit{grapheme, phoneme}) and use it to create a language model for phoneme recognition.

\section{Thesis overview}
